Die vorliegende Ausarbeitung hat den physischen Design-Flow von integrierten Schaltungen sowie die Rolle von Cadence Innovus und dem EDE-Tool beleuchtet. Dabei wurde gezeigt, wie die einzelnen Phasen – vom Floorplanning über Placement und Clock Tree Synthesis bis hin zu Routing und Signoff – ineinandergreifen und durch moderne EDA-Werkzeuge automatisiert werden können. Innovus bietet in Kombination mit dem EDE-Framework eine leistungsfähige Plattform, um komplexe Designs effizient umzusetzen und die Qualität durch konsistente Prozesse sicherzustellen.

Gleichzeitig ist zu betonen, dass diese Arbeit bewusst auf einer hohen Abstraktionsebene geblieben ist. Detaillierte Aspekte wie die zugrunde liegenden Optimierungsalgorithmen, komplexe Timing- und Power-Analysen sowie die vollständige Parametrisierung der Tool-Kommandos wurden nicht behandelt. Ebenso fehlen praktische Ergebnisse oder Benchmarks, da der Fokus auf dem konzeptionellen Verständnis des Design-Flows lag.

Für eine weiterführende Vertiefung empfiehlt sich die Auseinandersetzung mit den Optimierungsstrategien innerhalb der einzelnen Phasen, insbesondere Timing-Closure, Low-Power-Methoden und Multi-Corner-Analysen. Darüber hinaus bieten Themen wie Design-for-Test, ECO-Flows und der Einsatz von KI-gestützten Tools (z.\,B. Cadence Cerebrus) weitere Perspektiven für die Zukunft. Eine praktische Arbeit mit realen Projektdaten und die Teilnahme an spezialisierten Schulungen sind sinnvolle nächste Schritte, um das hier vermittelte Grundlagenwissen in die Praxis zu übertragen.