\documentclass[a4paper,11pt]{scrartcl}
 
\usepackage[ngerman]{babel}
\usepackage{graphicx}
\usepackage{siunitx}
\usepackage{subcaption}

\usepackage{float}
\usepackage{enumitem}

\usepackage{booktabs}

\usepackage{tikz}
\usetikzlibrary{arrows.meta,positioning,shapes,fit,calc}

\usepackage{pgfplots}
\pgfplotsset{compat=1.18}
\usepackage{pgfplotstable}

\usepackage[backend=biber,style=authoryear]{biblatex}
\addbibresource{./literatur/quellen.bib}

\newcommand{\zB}{z.\,B.\,}

\title{Vom RTL zum GDS}
\subtitle{ Einführung in das VLSI-Design von ICs mit Fokus auf Cadence Innovus}
\author{Fatih Mollamehmetoglu}
\date{\today} 

\pagestyle{headings}
 
\begin{document}


\maketitle
\thispagestyle{empty} 
\newpage

\tableofcontents
\newpage    

%glossar kommt eventuell später rein, ich überlege noch ob das sinnvoll ist
%\section{Glossar und Abkürzungen}
%
%RTL 
VHDL
EDA
GDSII
IP/Hardmacro
%\newpage

\section{Motivation und Zielsetzung}
Das Ziel dieser Ausarbeitung ist es, 
Studierenden der Elektro- und Informationstechnik, Informatik sowie verwandter Fachrichtungen einen verständlichen Einblick
in die Welt des digitalen Designs integrierter Schaltungen zu geben.  
Dabei soll ein grundlegendes Verständnis für die Entwicklung moderner Technologien vermittelt werden,
sodass die Leserinnen und Leser in der Lage sind,
sich anschließend eigenständig weiter in die Materie einzuarbeiten.  

Die Relevanz dieses Themas ergibt sich aus der wachsenden Komplexität heutiger integrierter Schaltungen 
und der zentralen Rolle des VLSI-Designs in nahezu allen Bereichen moderner Elektronik. 
Ein praxisnaher Bezug wird durch die Darstellung des spezifischen Design-Flows innerhalb der Firma REE (Renesas Electronics Europe) geschaffen.
Dies soll den Einstieg in reale industrielle Prozesse erleichtern und eine Grundlage für weiterführende Projekte oder
Tätigkeiten bieten.
\newpage

\section{Einführung in das Thema}

\subsection{Grundlagen des IC- und VLSI-Designs}

\begin{figure}[h!]
    \centering
     \includegraphics[width=0.5\textwidth]{figures/4004.jpg}
    \caption{erster digital entwickelter IC, Intel 4004}
    \label{Intel 4004}
\end{figure}

Integrierte Schaltungen (Integrated Circuits, ICs) sind Grundlage moderner Elektronik,
 welche uns überall im Alltag begegnet. 
 Sie ermöglichen die Funktionalität von Computern, 
 Smartphones und einer Vielzahl weiterer Systeme. 
 Ein IC ist hierbei im Wesentlichen eine Ansammlung von elektronischen Bauelementen - wie Transistoren, Widerständen und Kondensatoren - 
 die auf einem einzigen Halbleiterchip integriert sind. 
 Eben diese Integration erlaubt eine kompakte Bauweise, 
 hohe Leistungsfähigkeit und geringe Kosten.

Mit der zunehmenden Miniaturisierung und steigenden Anforderungen an Rechenleistung entwickelte sich das VLSI-Design (Very Large-Scale Integration) zu einer Schlüssel\-technologie. VLSI bezeichnet die Integration von Millionen bis Milliarden Transistoren auf einem einzigen Chip. Diese hohe Integrationsdichte ermöglicht nicht nur leistungsstarke Prozessoren und Speicher, sondern auch energieeffiziente Lösungen für mobile und eingebettete Systeme. Die Komplexität solcher Designs erfordert den Einsatz spezialisierter Methoden und Werkzeuge, um sowohl Funktionalität als auch Zuverlässigkeit und Herstellbarkeit sicherzustellen.

Die Entwicklung integrierter Schaltungen erfolgt heute in einem strukturierten Design-Flow, der sowohl Hardwarebeschreibung als auch automatisierte Werkzeuge umfasst. Typischerweise beginnt der Prozess mit der Spezifikation der gewünschten Funktionalität, gefolgt von der Beschreibung in Hardwarebeschreibungssprachen wie VHDL oder Verilog. Anschließend wird das Design durch Synthese in eine Netzliste überführt, die die logische Struktur der Schaltung beschreibt.

Darauf folgt die Place-and-Route-Phase, in der die logischen Elemente physisch auf dem Chip angeordnet und die Verbindungen hergestellt werden. Moderne Tools berück\-sichtigen dabei Aspekte wie Timing, Leistungsaufnahme und Flächenoptimierung. Nach umfangreichen Verifikations- und Signoff-Schritten (Simulation, Timing-Analyse, DRC und LVS) wird das finale Layout als GDSII-Datei an die Fertigung übergeben.

Dieser Flow wird heute stark durch EDA-Tools (Electronic Design Automation) und Methoden wie Design-for-Test, Low-Power-Optimierung sowie IP-Reuse unterstützt, um die Komplexität beherrschbar zu machen und die Time-to-Market zu verkürzen.

\subsection{Startpunkt des Design-Flows}

Wie in Abbildung~\ref{Design-Flow} dargestellt, lässt sich der gesamte IC-Designprozess grob in zwei Hauptbereiche unterteilen: das \textbf{Frontend} und das \textbf{Backend}.  
Das Frontend umfasst die frühen Phasen wie Systemspezifikation, Architekturdesign sowie das funktionale und logische Design. Hier wird die gewünschte Funktionalität des Chips beschrieben und in einer Hardwarebeschreibungssprache (z.\,B. VHDL oder Verilog) modelliert. Anschließend erfolgt die \textbf{Synthese}, bei der diese abstrakte Beschreibung in eine Gate-Level-Netzliste überführt wird. Diese Netzliste bildet die logische Struktur der Schaltung auf Basis standardisierter Zellen aus der verwendeten Technologie-Bibliothek.

Das Backend beginnt dort, wo die logische Beschreibung vorliegt und in eine physische Implementierung überführt werden muss. Der Ausgangspunkt unseres Backend-Designprozesses ist daher eine vollständig verifizierte Gate-Level-Netzliste, die aus der Synthesephase stammt. Sie enthält alle funktionalen Informationen und die logischen Verbindungen, jedoch noch keine physische Anordnung der Komponenten auf dem Chip.

Neben der Netzliste liegen zu diesem Zeitpunkt auch die relevanten Design-Constraints vor. Dazu gehören unter anderem Timing-Anforderungen, Vorgaben zur Leistungsaufnahme sowie Flächenbeschränkungen. Diese Randbedingungen sind entscheidend, um die physische Implementierung so zu gestalten, dass die funktionalen und technologischen Spezifikationen eingehalten werden.

Damit ist die Ausgangslage klar definiert: eine funktional korrekte, technologiegebundene Netzliste sowie die zugehörigen Constraints, die im weiteren Verlauf des Backend-Flows berücksichtigt werden müssen.

\begin{figure}[h!]
    \centering
    \begin{tikzpicture}[
        scale=0.7,
        transform shape,
        node distance=0.5cm,
        >=Stealth,
        rednode/.style={
            rectangle, draw=red!60, fill=red!5, very thick,
            minimum width=5cm, minimum height=1cm
        },
        bluenode/.style={
            rectangle, draw=blue!60, fill=blue!5, very thick,
            minimum width=5cm, minimum height=1cm
        },
        smallbluenode/.style={
            rectangle, draw=blue!60, fill=blue!5, very thick,
            minimum width=5cm
        },
        whitenode/.style={
            rectangle, draw=black!60, fill=white!5, very thick,
            minimum width=5cm, minimum height=1cm
        }
    ]
        \node[rednode, align=center] (system) {Systemspezifikation \\ und Architekturdesign};
        \node[rednode, below=of system, align=center] (func) {Funktionales Design \\ und Logisches Design};
        \node[bluenode, below=of func] (circuit) {Synthese};
        \node[bluenode, below=of circuit, align=center] (physical) {Physikalische \\ Implementierung};
        \node[whitenode, below=of physical, align=center] (verify) {Physikalische Verifikation\\ und Signoff};
        \node[whitenode, below=of verify] (fabrication) {Fabrikation};

        % Right column nodes
        %\node[smallbluenode, right=of func, xshift=3cm] (partitioning) {Partitioning};
        %\node[smallbluenode, below=of partitioning] (planning) {Floorplanning};
        %\node[smallbluenode, right=of circuit, xshift=3cm, yshift=2cm] (planning){Design Planning};
        %\node[smallbluenode, below=of planning] (placement) {Placement};
        %\node[smallbluenode, below=of placement] (clock) {Clock-Tree-Synthese};
        %\node[smallbluenode, below=of clock] (routing) {Routing};
        %\node[smallbluenode, below=of routing] (timing) {Timing Closure};
        \node[smallbluenode, right=of physical, xshift=3cm] (clock) {Clock-Tree-Synthese};
        \node[smallbluenode, above=of clock] (placement) {Placement};
        \node[smallbluenode, above=of placement] (planning){Design Planning};
        \node[smallbluenode, below=of clock] (routing) {Routing};
        \node[smallbluenode, below=of routing] (timing) {Signoff};

        % Connections
        \draw[->] (system) -- (func);
        \draw[->] (func) -- (circuit);
        \draw[->] (circuit) -- (physical);
        \draw[->] (physical) -- (verify);
        \draw[->] (verify) -- (fabrication);

        %\draw[dashed, -] (physical.north east) -- (partitioning.north west);
        \draw[dashed, -] (physical.north east) -- (planning.north west);
        \draw[dashed, -] (physical.south east) -- (timing.south west);
        %\draw[->] (partitioning) -- (planning);
        \draw[->] (planning) -- (placement);
        \draw[->] (placement) -- (clock);
        \draw[->] (clock) -- (routing);
        \draw[->] (routing) -- (timing);
    \end{tikzpicture}
    \caption{Allgemeiner Design-Flow beim IC-Design}
    \label{Design-Flow}
\end{figure}


\section{Physischer IC-Designflow}

\subsection{Floorplaning}
\begin{figure}[h!]
    \centering
    \begin{subfigure}[b]{0.48\textwidth}
        \centering
        \includegraphics[width=\textwidth]{figures/init.png}
        \caption{Chip-Layout nach dem Floorplanning}
        \label{fig:floorplan}
    \end{subfigure}
    \hfill
    \begin{subfigure}[b]{0.48\textwidth}
        \centering
        \includegraphics[width=\textwidth]{figures/powerrouting-fein.png}
        \caption{Detailansicht des Power-Grid-Designs}
        \label{fig:powergrid}
    \end{subfigure}
    \caption{Darstellung des Floorplanning und des Power-Grid-Designs}
    \label{fig:floorplan-powergrid}
\end{figure}

Nachdem die Ausgangslage mit einer verifizierten Gate-Level-Netzliste und den zugehörigen Design-Constraints definiert ist, beginnt die physische Implementierung mit dem \textbf{Floorplanning}.  
Beim Floorplanning wird die grundlegende Struktur des Chips festgelegt. Ziel ist es, eine effiziente Anordnung der funktionalen Blöcke, Makrozellen und Standardzellen zu planen, sodass spätere Schritte wie Platzierung und Verdrahtung optimal durchgeführt werden können.

Wichtige Aspekte beim Floorplanning sind:
\begin{itemize}
    \item \textbf{Chipgröße und Form}: Die Dimensionen müssen den Flächenvorgaben entsprechen und gleichzeitig genügend Platz für alle Komponenten sowie die Verdrahtung bieten.
    \item \textbf{Makroplatzierung}: Große Blöcke wie Speicher oder IP-Cores werden strategisch positioniert, um kurze Signalwege und gute Timing-Eigenschaften zu gewährleisten.
    \item \textbf{Power-Grid-Design}: Die Stromversorgung wird früh geplant, um eine stabile Versorgung aller Bereiche sicherzustellen.
    \item \textbf{I/O-Pads und Schnittstellen}: Die Positionierung der Ein- und Ausgänge beeinflusst die Signalführung und die Integration ins Gehäuse.
\end{itemize}

Abbildung~\ref{fig:floorplan} zeigt ein Design nach dem Floorplanning: Die IO-Pads sind entlang des Chiprandes angeordnet, während große IP-Blöcke und Makrozellen im Inneren platziert sind. Freie Bereiche dienen später der Platzierung von Standardzellen.  
Eine weitere Detailansicht in Abbildung~\ref{fig:powergrid} verdeutlicht das Power-Grid-Design, das für eine stabile Stromversorgung sorgt. Hier sind die VDD- und VSS-Leitungen sowie die zugehörigen Strukturen zu erkennen, die bereits in dieser frühen Phase berücksichtigt werden müssen. Diese sind über das gesamte Design verteilt in einem Grid angeordnet, aber aufgrund von Übersichtsgründen in der Abbildung~\ref{fig:floorplan} ausgeblendet.


\subsection{Placement}

"Lorem ipsum dolor sit amet, consectetur adipiscing elit, sed do eiusmod tempor incididunt ut labore et dolore magna aliqua. Ut enim ad minim veniam, quis nostrud exercitation ullamco laboris nisi ut aliquip ex ea commodo consequat. Duis aute irure dolor in reprehenderit in voluptate velit esse cillum dolore eu fugiat nulla pariatur. Excepteur sint occaecat cupidatat non proident, sunt in culpa qui officia deserunt mollit anim id est laborum.

\begin{figure}[h!]
    \centering
     \includegraphics[width=0.35\textwidth]{figures/place.png}
    \caption{Design nach Placement}
    \label{Placement}
\end{figure}

\subsection{CTS}

"Lorem ipsum dolor sit amet, consectetur adipiscing elit, sed do eiusmod tempor incididunt ut labore et dolore magna aliqua. Ut enim ad minim veniam, quis nostrud exercitation ullamco laboris nisi ut aliquip ex ea commodo consequat. Duis aute irure dolor in reprehenderit in voluptate velit esse cillum dolore eu fugiat nulla pariatur. Excepteur sint occaecat cupidatat non proident, sunt in culpa qui officia deserunt mollit anim id est laborum.

\begin{figure}[h!]
    \centering
     \includegraphics[width=0.35\textwidth]{figures/clock_tree.png}
    \caption{Synthetisierter Clocktree}
    \label{CTS}
\end{figure}

\subsection{Routing}

"Lorem ipsum dolor sit amet, consectetur adipiscing elit, sed do eiusmod tempor incididunt ut labore et dolore magna aliqua. Ut enim ad minim veniam, quis nostrud exercitation ullamco laboris nisi ut aliquip ex ea commodo consequat. Duis aute irure dolor in reprehenderit in voluptate velit esse cillum dolore eu fugiat nulla pariatur. Excepteur sint occaecat cupidatat non proident, sunt in culpa qui officia deserunt mollit anim id est laborum.

\begin{figure}[h!]
    \centering
     \includegraphics[width=0.35\textwidth]{figures/init.png}
    \caption{Design nach Floorplanning und Powerrouting}
    \label{Routing}
\end{figure}

\subsection{Signoff}

"Lorem ipsum dolor sit amet, consectetur adipiscing elit, sed do eiusmod tempor incididunt ut labore et dolore magna aliqua. Ut enim ad minim veniam, quis nostrud exercitation ullamco laboris nisi ut aliquip ex ea commodo consequat. Duis aute irure dolor in reprehenderit in voluptate velit esse cillum dolore eu fugiat nulla pariatur. Excepteur sint occaecat cupidatat non proident, sunt in culpa qui officia deserunt mollit anim id est laborum.

\begin{figure}[h!]
    \centering
     \includegraphics[width=0.35\textwidth]{figures/init.png}
    \caption{Design nach Floorplanning und Powerrouting}
    \label{Signoff}
\end{figure}
\newpage

\section{Cadence Innovus im Design-Flow}
\subsection{Überblick über Cadence-Innovus}

Cadence Innovus ist ein Electronic Design Automation (EDA)-Tool, das speziell für die physische Implementierung von integrierten Schaltungen entwickelt wurde. Es ist Teil der Cadence Digital Implementation Suite und wird in der Industrie als Standardlösung für den Backend-Design-Flow eingesetzt. Innovus deckt alle wesentlichen Schritte der physischen Designphase ab, darunter Floorplanning, Placement, Clock Tree Synthesis (CTS), Routing und Signoff.

Das Tool bietet leistungsfähige Optimierungsalgorithmen, die darauf ausgelegt sind, Timing-Anforderungen einzuhalten, die Leistungsaufnahme zu reduzieren und die Chip\-flä\-che effizient zu nutzen. Innovus ist für Designs mit sehr hoher Komplexität geeignet, die Millionen bis Milliarden von Standardzellen umfassen. Neben einer grafischen Benutzeroberfläche (GUI) unterstützt Innovus auch eine skriptbasierte Steuerung über Tcl, was eine flexible Automatisierung des Design-Flows ermöglicht.

Ein weiterer Vorteil ist die enge Integration mit anderen Cadence-Produkten wie Genus für die logische Synthese und Tempus für die Timing-Analyse. Diese Kombination erlaubt einen durchgängigen und konsistenten Design-Flow von der RTL-Beschreibung bis zum finalen Layout. Innovus wird aufgrund seiner Skalierbarkeit, Effizienz und umfangreichen Funktionen in der Industrie breit eingesetzt und gilt als eines der führenden Werkzeuge für die physische Implementierung moderner ICs~\parencite{cadence_innovus_ds}.

\subsection{EDE-Tool: Rolle und Integration}

Nachdem die grundlegenden Funktionen und die Bedeutung von Cadence Innovus im physischen Design-Flow erläutert wurden, stellt sich die Frage, wie die Interaktion zwischen den verschiedenen Werkzeugen und Prozessen effizient gestaltet werden kann. Hier kommt das EDE-Tool ins Spiel. Es bildet die Schnittstelle zwischen den Design-Teams und den eingesetzten EDA-Werkzeugen und sorgt für eine konsistente, automatisierte und kontrollierte Umgebung.

Das EDE-Tool ist eine GUI-basierte Layout-Design-Umgebung, die einen hohen Grad an Automatisierung bietet. Es dient als zentrale Plattform, um den gesamten Place-and-Route-Flow effizient zu steuern und konsistent zu halten. Alle technologie-, bibliotheks- und signoff-relevanten Einstellungen werden zentral gepflegt und sind für alle Nutzer einheitlich verfügbar, was die Qualität und Reproduzierbarkeit des Designs sicherstellt.
Innerhalb des EDE können einzelne Aufgaben des Design-Flows entweder als separate Schritte ausgeführt oder zu einem vollständigen beziehungsweise teilweisen PnR-Flow zusammengefasst werden. Für jede Designpartition steht ein eigenes „EDE-Cockpit“ zur Verfügung, das die Steuerung und Überwachung der jeweiligen Prozesse ermöglicht. Dabei können verpflichtende und optionale Schritte flexibel gewählt werden. Einige dieser Schritte müssen in einer festgelegten Reihenfolge ausgeführt werden, während andere unabhängig voneinander gestartet werden können.
Durch diese Struktur bietet das EDE-Tool nicht nur eine benutzerfreundliche Oberfläche, sondern auch eine robuste Grundlage für die Automatisierung komplexer Abläufe. Dies reduziert manuelle Fehler, beschleunigt den Designprozess und erleichtert die Zusammenarbeit in großen Projekten.

Abbildung~\ref{fig:ede_gui} zeigt die grafische Benutzeroberfläche des EDE-Tools. Auf der linken Seite sind die einzelnen Schritte des physischen Design-Flows dargestellt, die innerhalb des Tools ausgeführt werden können, wie beispielsweise \textbf{PreCTS (Initialisierung)}, \textbf{Routing}, \textbf{Signoff} und weitere Phasen. Diese Schritte können einzeln oder in Kombination gestartet werden. Auf der rechten Seite sind die zugehörigen Skripte sichtbar, die in den jeweiligen Phasen verwendet werden. Sie enthalten die spezifischen Befehle und Parameter, die den Ablauf der einzelnen Schritte steuern. Auf die Inhalte dieser Skripte sowie die wichtigsten Befehle wird im nächsten Abschnitt detailliert eingegangen.

\begin{figure}[H]
    \centering
    \includegraphics[width=\textwidth]{./figures/ede-gui.png}
    \caption{EDE-GUI mit Flow-Schritten (links) und zugehörigen Skripten (rechts)}
    \label{fig:ede_gui}
\end{figure}

\subsection{Implementierung der Designphasen im Innovus}

Die Steuerung des physischen Design-Flows erfolgt in unserer Umgebung über das EDE-Cockpit und phasenbezogene Skripte, die Cadence Innovus (und ggf. angebundene Tools) automatisiert ansteuern. Jede Phase lädt zunächst allgemeine Einstellungen über \emph{Always-Source}-Blöcke (Technologie-, Timing- und Constraint-Setup), bevor die spezifischen Schritte der Phase ausgeführt werden. Im Folgenden werden die wesentlichen Tätigkeiten pro Phase zusammengefasst; die zugehörigen Befehle/Steps sind in der Tabelle am Ende dieses Abschnitts aufgeführt.

\paragraph{Initialisierung (init)}%
Die Initialisierung bereitet das Design auf die physische Implementierung vor.
\begin{itemize}
    \item Allgemeines Projekt-Setup: Pfade, Variablen und Skript-Bindungen; Innovus-\-Session wird konfiguriert.
    \item Einlesen der Gate-Level-Netzliste und ggf. vorhandener \texttt{DEF}-Informationen (Start-Floorplan, Pins).
    \item Verbinden der Power-Netze (\texttt{VDD}/\texttt{VSS}) und Laden technologie-/bibliotheksrelevanter Dateien.
    \item Einlesen der RAM-/Memory-Instanzen und zugehöriger Definitionen.
    \item Einfügen von Endcaps und Well-Taps zur Gewährleistung korrekter Randbedingungen und Substratanbindung.
    \item Erstellen der Power-Routing-Struktur (Power-Grid); Festlegen von \texttt{pg\_keepout}-Zonen für Versorgung.
    \item Hinzufügen von \emph{respin cells} (falls projektspezifisch erforderlich).
    \item Koordinaten-Normalisierung, z.\,B. \emph{Shift Origin} auf die linke untere Ecke (\emph{LL}).
\end{itemize}

\paragraph{Placement (place)}%
In der Platzierungsphase werden logische Zellen positioniert und der Kontext für nachfolgende Optimierungen geschaffen
\begin{itemize}
    \item Laden der \emph{Always-Source}-Einstellungen: u.\,a. OCV-Modelle, globale Constraints, \texttt{dontTouch}-Sets
    \item Einstellen von  \emph{NDRs} (Non-Default Rules) für kritische Netze
    \item Festlegen von \emph{VT-Gruppen} gemäß UPC-Definitionen (Leckstrom/Timing-Zielsetzung)
    \item DFT-Vorplatzierung: Einfügen von \emph{OCCs}, Haupt-DFT-Bereich; Vorplatzierung von \emph{SIBs}, Isolationsmodulen und \emph{MBIST}-Gruppen
    \item Erneutes Setzen von \texttt{pg\_keepout}-Zonen (Versorgungsnetz-Reserven)
    \item Einlesen der \emph{Scan}-Definitionen; Schutz abuttierter Partition-Pins; Hinzufügen von Interface-Buffern
    \item \emph{Cell Padding} für Register (Abstandsregeln zur Entzerrung)
    \item (Re-)Connect der Power-Netze, falls erforderlich; \emph{cgate}-Platzierung (Clock-Gating-Zellen)
    \item Definition von \emph{Path Groups} für die Timing-Analyse und Optimierungspriorisierung
    \item Platzierung und kombinierte Optimierung via \texttt{place\_opt\_design}
    \item \emph{save design}
    \item Abschätzung der HPWL-Drahtlänge:
        \begin{equation}
        \text{HPWL} = (x_{\max} - x_{\min}) + (y_{\max} - y_{\min})
        \label{eq:hpwl}
\end{equation}
\end{itemize}

\paragraph{Clock Tree Synthesis (CTS)}%
Die CTS erzeugt und optimiert den physikalischen Taktbaum
\begin{itemize}
    \item Laden der \emph{Always-Source}-Einstellungen (CTS-spezifische Parameter)
    \item \texttt{ccopt} Konfiguration: Clock-Domain-Spezifikation, Puffer-/Inverter-Auswahl, Skew-/Latency-Ziele
    \item Ausführen der Taktbaum-Erzeugung und -Optimierung mit \texttt{clock\_opt\_design}
    \item Entfernen vorher gesetzter \emph{cell padding} (sofern nicht mehr erforderlich)
    \item \emph{Hold}-Optimierung zur Einhaltung kurzer Pfade (Setup/Hold-Balance)
    \item \emph{save design}
    \item Bestimmung der effektiven Clock-Latenz nach der Formel ~\eqref{eq:latency}
        \begin{equation}
        \text{Latency}_{\text{eff}} = \text{Latency}_{\text{src}} + \text{Latency}_{\text{net}}
        \label{eq:latency}
        \end{equation}

\end{itemize}

\paragraph{Routing (route)}%
Das Routing verbindet die Netze über die verfügbaren Metall-Layer unter Einhaltung der Designregeln
\begin{itemize}
    \item Laden der \emph{Always-Source}-Einstellungen; Setzen eines Genauigkeit/Laufzeit-Trade-offs (Routing-Strategie)
    \item (Re-)Connect der Power-Netze; Feinanpassung der \emph{NDRs} für ausgewählte Netze
    \item Kombination aus globalem und detailliertem Routing über \texttt{route\_opt\_design}
    \item \emph{ADDCAP}: Kapazitive Ergänzungen/Decaps zur Stabilisierung der Versorgung und Reduktion von IR-Drop/Noise
    \item \emph{save design}
\end{itemize}

\paragraph{Signoff (signoff)}%
Die Signoff-Phase validiert Herstellbarkeit, Timing und Zuverlässigkeit und bereitet die Übergabe an die Fertigung vor
\begin{itemize}
    \item RC-Extraktion (\texttt{extract\_rc}) für akkurate parasitäre Modelle
    \item \emph{Static Timing Analysis} und Abschluss-Timing (\texttt{signoff\_timing}); ggf. ECO-Schleifen
    \item Design Rule Checks (\texttt{signoff\_dr}) und ergänzende DFM-Prüfungen
    \item Optional: Power-/EM-Analysen (IR-Drop, Elektromigration) je nach Projekt-Setup
    \item Export der finalen Daten (z.\,B. \texttt{write\_gds} für GDSII)
\end{itemize}

\begin{table}[H]
    \centering
    \caption{Zuordnung zentraler Befehle zu den Phasen}
    \label{tab:phase_cmds}
    \begin{tabular}{p{0.25\textwidth} p{0.65\textwidth}}
        \toprule
        \textbf{Phase} & \textbf{Wichtige Befehle} \\
        \midrule
        Initialisierung (init) & 
        \texttt{read\_netlist}, \texttt{read\_def}, \texttt{create\_power\_nets}, \texttt{pg\_keepout} \\
        \midrule
        Placement (place) & 
        \texttt{place\_opt\_design}, \texttt{pg\_keepout} \\
        \midrule
        CTS & 
        \texttt{clock\_opt\_design} \\
        \midrule
        Routing (route) & 
        \texttt{route\_opt\_design} \\
        \midrule
        Signoff (signoff) & 
        \texttt{<placeholder>} \\
        \bottomrule
    \end{tabular}
\end{table}
\newpage

\section{Fazit und Ausblick}
Die vorliegende Ausarbeitung hat den physischen Design-Flow von integrierten Schaltungen sowie die Rolle von Cadence Innovus und dem EDE-Tool beleuchtet. Dabei wurde gezeigt, wie die einzelnen Phasen – vom Floorplanning über Placement und Clock Tree Synthesis bis hin zu Routing und Signoff – ineinandergreifen und durch moderne EDA-Werkzeuge automatisiert werden können. Innovus bietet in Kombination mit dem EDE-Framework eine leistungsfähige Plattform, um komplexe Designs effizient umzusetzen und die Qualität durch konsistente Prozesse sicherzustellen.

Gleichzeitig ist zu betonen, dass diese Arbeit bewusst auf einer hohen Abstraktionsebene geblieben ist. Detaillierte Aspekte wie die zugrunde liegenden Optimierungsalgorithmen, komplexe Timing- und Power-Analysen sowie die vollständige Parametrisierung der Tool-Kommandos wurden nicht behandelt. Ebenso fehlen praktische Ergebnisse oder Benchmarks, da der Fokus auf dem konzeptionellen Verständnis des Design-Flows lag.

Für eine weiterführende Vertiefung empfiehlt sich die Auseinandersetzung mit den Optimierungsstrategien innerhalb der einzelnen Phasen, insbesondere Timing-Closure, Low-Power-Methoden und Multi-Corner-Analysen. Darüber hinaus bieten Themen wie Design-for-Test, ECO-Flows und der Einsatz von KI-gestützten Tools (z.\,B. Cadence Cerebrus) weitere Perspektiven für die Zukunft. Eine praktische Arbeit mit realen Projektdaten und die Teilnahme an spezialisierten Schulungen sind sinnvolle nächste Schritte, um das hier vermittelte Grundlagenwissen in die Praxis zu übertragen.


\printbibliography

\end{document}
