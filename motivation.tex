Das Ziel dieser Ausarbeitung ist es, 
Studierenden der Elektro- und Informationstechnik, Informatik sowie verwandter Fachrichtungen einen verständlichen Einblick
in die Welt des digitalen Designs integrierter Schaltungen zu geben.  
Dabei soll ein grundlegendes Verständnis für die Entwicklung moderner Technologien vermittelt werden,
sodass die Leserinnen und Leser in der Lage sind,
sich anschließend eigenständig weiter in die Materie einzuarbeiten.  

Die Relevanz dieses Themas ergibt sich aus der wachsenden Komplexität heutiger integrierter Schaltungen 
und der zentralen Rolle des VLSI-Designs in nahezu allen Bereichen moderner Elektronik. 
Ein praxisnaher Bezug wird durch die Darstellung des spezifischen Design-Flows innerhalb der Firma REE (Renesas Electronics Europe) geschaffen.
Dies soll den Einstieg in reale industrielle Prozesse erleichtern und eine Grundlage für weiterführende Projekte oder
Tätigkeiten bieten.

\begin{figure}[h!]
  \centering
  \begin{tikzpicture}
    \begin{axis}[
      width=0.9\linewidth,
      height=10cm,
      ymode=log,
      log basis y=10,
      xlabel={Jahr},
      ylabel={Transistoren pro Chip},
      ymin=1e3, ymax=1e11,
      grid=both,
      grid style={dotted,gray!50},
      legend pos=south east,
      every axis plot/.append style={thick}
    ]
      % Punkte + Linie
      \addplot+[only marks, mark=*] table [x=Year, y=Transistors, col sep=comma] {./data/moore_raw_data.csv};
      \addlegendentry{Rohdaten}

      % Einfache Trendlinie (Moore ~ Verdopplung alle ~2 Jahre):
      % Beispiel: passe Startwert + Verdopplungsrate an dein Intervall an.
      \addplot+[domain=1970:2025, samples=20] {2300 * 2^((x-1971)/2)};
      \addlegendentry{Moore-Trend (Verdopplung $\approx$ 2 Jahre)}
    \end{axis}
  \end{tikzpicture}
  \caption{Moore’s Law – illustrative Rohdaten und Trend}
  \label{fig:moore}
\end{figure}