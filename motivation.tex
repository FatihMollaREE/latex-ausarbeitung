
Das Ziel dieser Ausarbeitung ist es, 
Studierenden der Elektro- und Informationstechnik, Informatik sowie verwandter Fachrichtungen einen verständlichen Einblick
in die Welt des digitalen Designs integrierter Schaltungen zu geben.  
Dabei soll ein grundlegendes Verständnis für die Entwicklung moderner Technologien vermittelt werden,
sodass die Leserinnen und Leser in der Lage sind,
sich anschließend eigenständig weiter in die Materie einzuarbeiten.  

Die Relevanz dieses Themas ergibt sich aus der wachsenden Komplexität heutiger integrierter Schaltungen 
und der zentralen Rolle des VLSI-Designs in nahezu allen Bereichen moderner Elektronik. 
Ein praxisnaher Bezug wird durch die Darstellung des spezifischen Design-Flows innerhalb der Firma REE (Renesas Electronics Europe) geschaffen.
Dies soll den Einstieg in reale industrielle Prozesse erleichtern und eine Grundlage für weiterführende Projekte oder
Tätigkeiten bieten.

\begin{figure}[h!]
  \vspace{0.5cm}
  \centering
  \begin{tikzpicture}
    \begin{axis}[
      width=0.8\linewidth,
      height=7cm,
      ymode=log,
      log basis y=10,
      xlabel={Jahr},
      ylabel={Transistoren pro Chip},
      ymin=1e3, ymax=1e11,
      grid=both,
      grid style={dotted,gray!50},
      legend style={
          font=\footnotesize,
          at={(0.98,0.02)},
          anchor=south east,
          draw=none
      },
      every axis plot/.append style={thick},
      ticklabel style={/pgf/number format/1000 sep={}}
    ]
      % Rohdatenpunkte aus CSV
      \addplot+[only marks, mark=*, mark size=1.5pt, blue] 
        table [x=Year, y=Transistors, col sep=comma, header=true] {./data/moore_raw_data.csv};
      \addlegendentry{Rohdaten}

      % Moore-Trendlinie nur gerade Linie
      \addplot+[red, thick, mark=none, domain=1970:2025, samples=2] 
        {2300 * 2^((x-1971)/2)};
      \addlegendentry{Moore-Trend}

    \end{axis}
  \end{tikzpicture}
  \caption{Mooresche Gesetz - Die exponentielle Zunahme der Transistordichte}
  \label{fig:moore}
\end{figure}


\subsection*{Berechnung der Trendlinie}

Die Trendlinie wird durch die exponentielle Ausgleichsgerade approximiert:

\needspace{7\baselineskip} 
\begin{align}
y(x) &= y_0 \cdot 2^{\frac{x - x_0}{T}} \label{eq:moore}\\
\intertext{mit}
y(x) &:\ \text{Anzahl der Transistoren im Jahr } x \notag\\
y_0 &:\ \text{Startwert der Transistoranzahl im Basisjahr } x_0 \notag\\
T   &:\ \text{Verdopplungszeit in Jahren (hier ca.\ 2 Jahre).} \notag
\end{align}